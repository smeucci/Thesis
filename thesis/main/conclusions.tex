\chapter*{Conclusions}
\addcontentsline{toc}{chapter}{Conclusions}
\chaptermark{Conclusions}

In this thesis, we have proposed an approach for forensic analysis of video file containers. Our method exploits the differences in the file containers structure and content from different manufacturers. In particular, we have dealt with Source Identification and Integrity Verification focused on videos taken from smartphones and tablets.

To implement our method we have developed two Java programs that can be used command line tool, \emph{Video Format Tool} that is used to extract and manipulate the file containers, the \emph{File Origin Analysis} that computes the likelihood that a query video belongs to a certain class of devices. Also, a web application was realized to allow users to easily upload query videos to test.

The experiments show good results for both the Source Identification and the Integrity Verification.

For the Integrity Verification, we have showed that using software editing tools on a video leaves traces on the file container by either adding new boxes or by changing the structure. Our system is always able to detect the changing in the file container structure of a minimally processed video when compared to a reference video. 

For Source Identification, we were able to correctly classify our test videos based on their file containers in most cases. The misclassifications are due to different manufacturers using a very similar structure for the file containers. Also, how the reference population is built during the training phase might cause some errors. It is possible that having a more varied dataset in terms of devices with a greater number of videos per device will lead to a more accurate classification.

Although we get good results, this thesis should be considered preliminary work whose main intent is to show that using file containers for forensic analysis of digital video is an interesting approach. In fact, there are many aspect that could be improved and investigated more:

\begin{itemize}

\item Some boxes are only present in the file containers when a particular condition happens during the acquisition of the video. Such is the case of the box \emph{xyz} that contains information about the GPS coordinates and that is located inside the User Data Box (\emph{udta}). This box is only present if, during the acquisition of the video, localization was active on the device. Given a file container, the presence or the absence of such box does not give us any information about the source device. However, adding the box \emph{xyz} means adding the box \emph{udta}, located inside the Movie Box (\emph{moov}). Since we are using an indexing to determine the position of a box, this addition causes all the box indexs to shift. So two videos taken from the same device, one using GPS location and the other not, have very different file container representation for our system that sees them as coming from distinct sources. Right now, this issue is resolve with a workaround that will ignore the box \emph{xyz}. However, there is the need to implement a more elegant and extensible solution that can deal with other unknown boxes that behave the same.

\item One aspect that could be improved is how the attributes values are compared. Right now some attributes that are related only to the specific video, such as \emph{creationTime} or \emph{duration}, are ignored. An improved solution might be implementing a comparator function that specializes in a different way accordingly with the kind of attribute under consideration. For example, comparing values for the attribute \emph{flag} is different from comparing value of the attribute \emph{creationTime}. For the first, it is sufficient to directly confront the values and check if they differ or not; for the latter, it might be a better idea to check whether the dates are in the same format. In this way, we remove the need to ignore certain attributes and we could also improve the performance of our approach because we would be able to access more information.

\item The dataset we used for experiments is limited in terms of number of videos but more specifically in terms of types of devices. It would be interesting to try our method when more devices from different brands and models are available. Moreover, for the videos in our dataset only the generic operating system is known. We do not possess information about the version of the operating systems for each video. Changing the version of an operating system could also mean changes in the file container. We could harness this information to allow us to be even more specific in identifying the source device for a query video.

\item The problem of how to create the reference population is still open. We notice that based on how the reference population is built, the discriminant powers of the attributes change and so changes the computed likelihood given a query video.
It might be possible to find a more systematic way to determine the reference population or even to developed a method that allow us to use the entire population as a reference without losing in performance.

\end{itemize}

