\chapter{State of the Art}

\section{Introduction to Multimedia Forensics}

With the increasing spread of digital audio and video content, the analysis of these multimedia objects is rapidly assuming importance in the context of digital investigations, which consider both digital data and digital devices.

In digital investigations, multimedia content such as images, audio, and video are more and more being used as forensic evidence. It is therefore very important to be able to extract information from such content in a reliable manner.

Multimedia Forensics has the aim to gain knowledge on a multimedia content life cycle exploiting the traces that the various processing steps leave on the data. In fact, the idea behind Multimedia Forensics is that each acquisition device and each processing operation on a digital resource leave on the media content data some traces, often called fingerprints, which characterize its history.

Many algorithms and techniques have been developed by the scientific community based on the extraction of features from the stream of audio-visual data of the multimedia content. By taking advantage of these features, such techniques try to infer information about the source/acquisition device and about which encoding and editing processes the digital resource was subject to during its life cycle. Specifically for multimedia content such as images and videos, the main approaches set themselves the goal to identify the source of digital resource and/or to determine if the content is authentic or has been modify from its original without any a priori information about the content under analysis. This is possible using just these features and tools that allow to check for the presence or absence of such features or fingerprints that are intrinsically linked to the multimedia data by the acquisition device, the enconding step and any post-processing or editing software tools. In fact, we can distinguish three types of traces left on a multimedia content: acquisition traces, encoding traces, and editing traces.

As explained, from a scientific point of view, research has produced a large number of techniques for the analysis of multimedia content. From the point of view of the application of such techniques in the courtroom, however, there is an important gap that it is probably due to a lack of communication between the legal part and the scientific part, as well as a not complete maturity of the techniques that are often based on results obtained in laboratory contexts a not in real-world scenarios.

In addition, there is the need for greater sharing of standard in the field of digital multimedia forensics investigations that aid Multimedia Forensics to grow and reach maturity.

Several communities and groups have worked to put together guidelines and standard on important aspects of digital investigations, such as the chain of custody, data authentication, application of the scientific method, documentation, and reporting.
The ISO/IEC JTC1 Working Group 4 is one of those groups that seek to give international standards {LISTA} whose main purpose is to promote the best procedures and methods for the investigation of digital evidence. It also encourage the adoption of approaches for the forensic analysis on multimedia content that are shared at an international level, in order to ease the comparison and the combination of results from different entities and organizations and also through various jurisdictions, so as to increase the reliability of such methods and of the results.

Another group that aims at giving standards and guidelines for the digital investigations is the Scientific Working Group on Digital Evidence (SWGE) that deals with get in contact different organizations that work in the field of Multimedia Forensics in order to promote communication and cooperation and to ensure higher quality and consistency within the forensic community.

The Scientific Working Group on Imaging Technologies (SWGIT), instead, focuses his work on image analysis technology and has the aim to facilitate the integration of such methods of analysis of images in the context of the judicial system by providing best practices and guidelines for the acquisition, storage, processing, analysis, transmission, output image and archive of digital evidence.

Regarding the forensic analysis of images and/or videos, the process is defined to be composed of three main tasks: technical preparation, examination, interpretation. The technical preparation is concerned with all those operations necessary to prepare videos and images to the other tasks. The examination is the main part of the forensic analysis and deals with the application of techniques that aim to extract information from images/videos. The interpretation concerns the analysis of digital content from experts in order to provide conclusions on the features extracted from the images/videos under examination.

In this context, it becomes essential the figure of the Forensic Analyst, i.e. one who is able to develop and apply these methods for the analysis of digital content, interpret the results and make a summary of the results from different techniques in order to increase the reliability of the conclusions. It is also able to perform all of the analysis tasks in accordance with the standards shared by the forensic community. The Forensic Analyst is able to find the traces left on the multimedia data and to acquire information on the object under examination such as who is the source/acquisition device, whether the content is authentic and if the resource is intact, and so on.

\subsection{Applications}

Le principali applicazione per l'analisi forense sono: la source idenfitication, l'autenticazione e l'integrità della risorsa multimediale.

Il processo di identificazione della sorgente ha come obbiettivo quello di recuperare informazioni riguardo il dispositivo sorgente del contenuto multimediali in esame. É possibile identificare la sorgente a vari livelli di dettaglio. Ad esempio distinguere fra tipi di sorgenti, distinguere fra differenti modelli di sorgente dello stesso tipo e distinguere fra dispositivi diverse ma dello stesso tipo e modello. Informazioni possono essere ottenuti dai metadati.

Il problema dell'autenticazione riguarda il compito di stabilire se il contenuto multimediale è un accurato rendiconto di un evento originale. Il processo di analisi si basa sulla ricerca di inconsistenze nelle features del segnale audio-visivo.

Il problema dell'integrità riguarda il compito di stabilire se un contenuto multimediale è stato modificato o meno da quando è stato acquisito dal dispositivo sorgente. L'analisi si basa sulla ricerca di tracce lasciate dai tools di editing e processing durante il ciclo di vita che non sono compatibili con i contenuti provenienti dal dispositivo sorgente che è noto.

\subsection{Tools}

Un file multimediali può essere visto come un pacco composto da due parti principali: l'header, che contiene i metadati ovvero informazioni sul contenuto del file; il contenuto stesso, ovvero il flusso di dati che forma il segnale audio-visivo.
In generale, l'estrazione di features si basa sull'analisi delle tracce lasciate sia sui dati veri e proprio che sui metadati del file multimediale.

Per quanto riguarda l'ispezione del flusso di dati, l'esaminazione consiste di due aspetti principali: l'interpretazione del contenuto, che riguarda l'analisi del contesto come comprendere cosa sta succedendo, chi sono i soggetti e gli oggetti coinvolti, informazioni sull'ambiente e tutte le informazione che derivano dall'osservazione umana; identificazione di dettagli rivelanti nella scena rappresentata, come anomalie audio-visive, features interessanti come direzione della luce, ombre, prospettiva incoerenze, segni di smudge ecc. Sempre nel contesto dell'ispezione del segnale audio-visivo, è l'analisi e l'enhancement dei contenuti, come migliorare il segnale per rilevare dettagli o oggetti rilevanti, estrazione di relazioni dimensionali come dimensioni di oggetti o soggetti, comparazione fotografica fra oggetti noti e oggetti rappresentati nella scena.

L'altro strumento riguarda l'estrazione e l'analisi di metadati. I metadati possono essere estratti facilmente e possono contenere molte inforamzioni riguardo il segnale audio-visivo come ad esempio (video) il dispositivo sorgente, lo spazio colore, la risoluzione, i parametri di compressione, la data, dati gps, frame rate, (audio) format tags, bit rate, sample rate, numero di canali. Ovviamente il tipo e il numero di metadati dipende dal tipo del file in esame e quali processi ha subito durante il suo ciclo di vita.Ad esempio, per quando riguarda le immagini i metadati sono estratti dall'header Image file format (Exif). Una volta ottenuti i metadati è compito dell'analista forense verificare la compatibilità, la completezza e la coerenza delle inforamzioni estratte rispetto allo scenario da cui il contenuto si suppone provenire: metadati saranno differenti se una risorsa proviene da un social network piuttosto che direttamente dal dispositivo di acquisizione.



\section{Forensic Analysis of Video File Container}

\subsection{Video File Containers}

Gli standard dei formati video prescrivono un numero limitato di features, lasciando quindi molta libertà alla casa produttrice. Ciò può essere sfruttato come risorsa per identificare e autenticare, dato un video, la sorgente di provenienza. In particolare lo studio si concentra su MP4-like video format (mp4 e mov).

In un mondo sempre più digitale c'è sempre più la necessità di metodi per l'autenticazione dei dati multimediali, del riconoscimento della loro sorgente di provenienza e della loro storia.

L'anilisi forense si è principalmente concentrata su l'analisi di immagini sviluppando tecniche basate su due principali approcci: analisi del rumore (PRNU); analisi dei metadati JPEG e EXIF.

Anche l'analisi forense di video digitale è sempre più rilevente. I primi lavori hanno seguito un procedimento simile a quello delle immagini, basandosi rumore e artefatti presenti nel video ed anche l'utilizzo di metadati JPEG e EXIF.
Tuttavia, riguardo all'utilizzo di metadati per l'autenticazione di sorgenti digitali, sono state espresse preoccupazioni rispetto falsicabilità. Infatti, è solo questione di trovare il giusto strumento, spesso disponibile pubblicamente, per modificare facilmente informazioni di alto livello come appunto i metadati EXIF.

Contrariamente, gli strumenti noti non permettono di modificare strutture core dei file multimediali, come appunto i video file containers, ed altri strumenti pubblicamente disponibili che permettano di falsificare sistematicamente i containers non sono noti. Quindi queste caratteristiche di basso livello riducono le preoccupazioni riguardo la falsificabilità di tali metadati in quanto solo in possesso di skills avanzate di programmazione sarebbe possibile modificare la struttura interna dei file video rendendo la falsificabilità un'operazione non triviale ed altamente tecnica.

Il il formato container MOV è stato introdotto da Apple nel 1991. Utilizzando questo formato come base, sono stati introdotti anche i formati MP4 e 3GP.

I file containers di questo formato sono composti da atomi (o boxes) che sono identificati caratteri unici di 4 byte, preceduti dalla dimensione dell'atomo. Tali atomi possono avere o meno dei campi e possono essere o meno nidificati, ovvero possono contenere altri atomi; quindi si incontrano 4 tipi di atomi:
- atomi senza campi né che contengono altri atomi;
- atomi senza campi che contengono altri atomi;
- atomi con campi ma che non contengono altri atomi;
- atomi con campi che contengono altri atomi;

La struttura più comune dei MP4-like containers è la seguente:
- atomo *ftyp*: è semi-obbligatorio, ovvero il più recente standard ISO prevede che sia presente e che venga esplicitato il prima possibile nel file container. Fa riferimento alla specifica del tipo di file con la quale il file video, a cui il container appartiene, è compatibile. I campi principali sono *majorbrand*, che indica quale specifica sia la migliore, e vari campi numerati chiamati *compatiblebrand*, che indica una specifica che è compatibile con il file video. Non contiene altri atomi.
- atomo *mdat*: contiene lo stream dei dati ed è accompagnato dal campo *size* che ne indica la dimensione. Non contiene altri atomi.
- atomo *moov*: è l'atomo con la struttura più complessa del container. É un atomo nificati, ovvero contiene numerosi altri atomi. In questo atomo, e nella sua sotto-struttura, sono contenuti i metadati necessari per la decodifica del flusso dati contenuto nell'atomo *mdat*.

Tale struttura non è sempre rispetta. I file container dei formati MP4-like sono molto complessi e presentano numerosi elementi e differenze dipendenti dalla sorgente e dalla casa manifatturiera; quindi costituiscono uno strumento prezioso per un'analista forense.
Le differenze dei containers di diversi file video si possono presentare nei seguenti modi:
- posizione relativi degli atomi: nonostante lo standard dia specifiche riguardo la posizione di determinati atomi (in particolare quelli principali), ciò non è sempre rispettato. Soprattutto, si ha variazione di posizione per quegli atomi la cui posizione non è specificata dallo standard. Questo discorso è valido sia al primo livello, ovvero quello contenente gli atomi *ftyp*, *mdat*, *moov*, sia per quanto riguarda gli atomi della sotto-struttura contenuta nell'atomo *moov*.
- presenza di atomi aggiuntivi non standard ai vari livelli del container.
- differenze nei valori dei campi contenuti negli atomi.

Tutte queste possibili differenze e tipi di differenze danno vita ad un numeroso insieme di combinazioni che l'analista forense può sfruttare per autenticare una risorsa video.

PARLARE ANCHE DEL FORMATO AVI.

\subsection{Applications for Video File Container}

Il problema dell'autenticazione si basa sull'analisi del segnale audio-visivo, quindi non è nel nostro dominio visto che usiamo i file container che sono metadati. Con i metadati non si può stabilire se il contenuto multimediale è una rappresentazione autentica di un evento originale.

Il problema dell'integrativa invece viene affrontato. Infatti, come descritto prima, i file container di un video è il file container creato dall'ultimo strumento del ciclo di vita. Se un video è modificato anche leggermente da un software avrà un suo particolare container. Questo container conterrà atomi aggiuntivi o in posizione diversa e valori degli attributi differenti rispetto ad un video proveniente direttamente dalla sorgente. É quindi possibile confrontare un container reference, preveniente da un dispositivo noto, ed un container query, che si suppone provenire da quel dispositivo noto, per verificare se il video è stato o meno modificato.

Il problema della source classification ha senso perchè siccome gli standard definiscono solo alcune features obbligatorie, lasciano molta libertà alle case manifatturiere. Significa che ogni classe di dispositivi avrà il suo container diverso. Ciò può essere usato per distinguere fra dispositivi, nel caso di video provenienti da smarthphone, dello stesso brand, dallo stesso brand e model, dallo stesso brand, model e sistema operativo.
Per fare ciò occorre rappresentare in qualche modo le varie classi presenti in un training set. Inoltre occorre una misura di compatibilità che stimi la probabilità di un container query di appartenere ad una determinata classe di dispositivi.