\chapter*{Abstract}
\addcontentsline{toc}{chapter}{Abstract}
\chaptermark{Abstract}

The investigation on digital contents is become more and more relevant with the increasing diffusion in our life of devices that can acquire images and videos. For this reason, there is the need to validate such digital contents so they can legitimately be used as evidence in a courtroom.
Research in the field of Multimedia Forensics have developed many techniques and methodologies that are able to verify the authenticity and the integrity of digital resources, such as images and videos, both analyzing the audio-video signal and the metadata information.
This thesis proposes an approach for forensic analysis of digital videos that uses information contained in file containers. In fact, video file format standards define only a limited set of mandatory features, leaving freedom of interpretation to the device manufacturers.
Information that comes from these difference in design decision is what our method exploits both with regard to Source Identification and Integrity Verification.

\chapter*{Sommario}
\addcontentsline{toc}{chapter}{Sommario}
\chaptermark{Sommario}

L'uso di contenuti digitali a fini investigativi sta diventando sempre più di attualità con l'aumentare della diffusione di dispositivi in grado di produrre immagini e video. Per tale motivo, è sempre più maggiore l'esigenza di validare tali contenuti digitali al fine di poter essere usati legittimamente come prove digitali.
La ricerca in ambito Multimedia Forensics si è adoperata nello studio di tecniche e metodologie che siano in grado di verificare l'autenticità e l'integrità delle risorse digitali, sia analizzandone il flusso dati che i metadati. 
Questa tesi propone una tecnica di analisi forense di video digitali che sfrutta i file container. Infatti, gli standard per i formati dei file video definiscono solo un numero limitato di features obbligatorie, lasciando così spazio per l'interpretazione da parte delle case produttrici di dispositivi. Queste differenze di interpretazione ed implementazione sono ciò che il nostro approccio sfrutta sia riguardo la Source Identification sia per l'Integrity Verification.