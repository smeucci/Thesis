\chapter*{Abstract}
\addcontentsline{toc}{chapter}{Abstract}
\chaptermark{Abstract}

L'uso di contenuti digitali come prove di tribunale sta diventando sempre più di attualità con l'aumentare nella diffusione di dispositivi in grado di produrre immagini e video. Per tale motivo, è sempre più crescente la richiesta di tecniche che siano in grado di verificare l'autenticità di tali contenuti digital in modo da poter essere legittimamente usati in tribunale.
Multimedia forensics si è adoperata nello studio di tecniche che siano in grado di verificare l'autenticità e l'integrità delle risorse digitali sia analizzando il flusso dati che i metadati. 
Questa tesi propone una tecnica di analisi forense di video digitali che sfrutta i file container. Infatti, gli standard per i formati dei file video definiscono solo un numero limitato di features obbligatorie, lasciando così spazio per l'interpretazione da parte delle case produttrici di dispositivi. Queste differenze di interpretazione ed implementazione sono ciò che il nostro approccio sfrutta sia riguardo la determinazione del dispositivo sorgente sia per la verifica dell'integrità.