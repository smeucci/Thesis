\chapter{Video File Containers}


struttura

The ISO Base Media File Format is designed to contain timed media information for a presentation in a
flexible, extensible format that facilitates interchange, management, editing, and presentation of the media.
This presentation may be ‘local’ to the system containing the presentation, or may be via a network or other
stream delivery mechanism.
The file structure is object-oriented; a file can be decomposed into constituent objects very simply, and the
structure of the objects inferred directly from their type.
The file format is designed to be independent of any particular network protocol while enabling efficient
support for them in general.
The ISO Base Media File Format is a base format for media file formats.

This International Standard specifies the ISO base media file format, which is a general format forming the
basis for a number of other more specific file formats. This format contains the timing, structure, and media
information for timed sequences of media data, such as audio/visual presentations.

Box:
An object-oriented building block defined by a unique type identifier and length (called ‘atom’ in some
specifications, including the first definition of MP4).

Presentation:
One or more motion sequences (q.v.), possibly combined with audio.

File Structure
Files are formed as a series of objects, called boxes in this specification. All data is contained in boxes; there
is no other data within the file. This includes any initial signature required by the specific file format.
All object-structured files conformant to this section of this specification (all Object-Structured files) shall
contain a File Type Box.


Object Structure
An object in this terminology is a box.
Boxes start with a header which gives both size and type. The header permits compact or extended size (32
or 64 bits) and compact or extended types (32 bits or full UUIDs). The standard boxes all use compact types
(32-bit) and most boxes will use the compact (32-bit) size. Typically only the Media Data Box(es) need the 64-
bit size.
The size is the entire size of the box, including the size and type header, fields, and all contained boxes. This
facilitates general parsing of the file.
The definitions of boxes are given in the syntax description language (SDL) defined in MPEG-4 (see reference
in clause 2). Comments in the code fragments in this specification indicate informative material.
The fields in the objects are stored with the most significant byte first, commonly known as network byte order
or big-endian format.

The semantics of these two fields are:
size is an integer that specifies the number of bytes in this box, including all its fields and contained
boxes; if size is 1 then the actual size is in the field largesize; if size is 0, then this box is the last
one in the file, and its contents extend to the end of the file (normally only used for a Media Data Box)

type identifies the box type; standard boxes use a compact type, which is normally four printable
characters, to permit ease of identification, and is shown so in the boxes below. User extensions use
an extended type; in this case, the type field is set to ‘uuid’.
Boxes with an unrecognized type shall be ignored and skipped.
Many objects also contain a version number and flags field:

The semantics of these two fields are:
version is an integer that specifies the version of this format of the box.
flags is a map of flags
Boxes with an unrecognized version shall be ignored and skipped.

Object Structure
The file is structured as a sequence of objects; some of these objects may contain other objects. The
sequence of objects in the file shall contain exactly one presentation metadata wrapper (the Movie Box). It is
usually close to the beginning or end of the file, to permit its easy location. The other objects found at this level
may be a File-Type box, Free Space Boxes, Movie Fragments, Meta-data, or Media Data Boxes.

