%\documentclass[titlepage]{book}
\documentclass[italiano,laurea,twoside,10pt]{style/thesis}
%%Packages:
\usepackage[utf8]{inputenc}
%\usepackage[UKenglish]{babel}
\usepackage{graphicx}
\usepackage{subcaption}
%\usepackage{subfigure}
\usepackage{cite}
\usepackage[hidelinks]{hyperref}
\usepackage{listings}
\usepackage{latexsym}
\usepackage{amsmath,amsthm,amssymb}
\usepackage{proof}
\usepackage{stmaryrd}
\usepackage{epstopdf}
\usepackage{pgf}
\usepackage{tikz}
%\usepackage{pgfplots}
\usepackage{setspace}
\usepackage{array}
\usepackage{ragged2e}

% \usepackage{algorithm}
\usepackage[noend]{style/algpseudocode}
\usepackage[ruled,noline,linesnumbered,algochapter]{algorithm2e}
\usetikzlibrary{automata,arrows,decorations.pathreplacing}
\usetikzlibrary{positioning}

%trick to avoid floating images trespassing sections boundaries
\usepackage[section]{placeins}

%table utils
\newcolumntype{C}[1]{>{\centering\arraybackslash}p{#1}}


%margins
% \usepackage{layout}
% \usepackage[inner=4cm,outer=2cm]{geometry}

%%Notation:

%%vectors
\renewcommand{\vec}[1]{\boldsymbol{#1}}
%%sets
\newcommand{\set}[1]{\mathcal{#1}}
%%matrixes
\newcommand{\mat}[1]{#1}
%%norm
\newcommand{\norm}[1]{\left\Vert #1 \right\Vert}
%%defeq
\newcommand{\defeq}{\triangleq}
%%
\newcommand{\net}[1]{\mathrm{#1}}
%% pair<x,y>
\newcommand{\pair}[2]{\langle#1,#2\rangle}


%%Envirorments:

% \newtheoremstyle{definition}{}{}{\itshape}{}{\bfseries}{.}{.5em}{\thmnote{#3's }#1}
\theoremstyle{definition}
\newtheorem{defn}{Definition}
\theoremstyle{definition}
\newtheorem{remark}{Remark}
\theoremstyle{definition}
\newtheorem{thm}{Theorem}

%Paths:
%\graphicspath{ {./images/} }

% Title Page
\title{Video Forensic based on File Format Analisys}
\author{Saverio Meucci}
\titolocorso{Ingegneria Informatica}
\degreeyear{2015/2016}
\date{}
\chair{Prof. Alessandro Piva\\ Prof. Fabrizio Argenti}
\othermembers{Dott. Marco Fontani\\ Dott. Massimo Iuliani}
\numberofmembers{2}

%%%%%%%%%%%%%%%%%%%%%%%%%%%%%%%%%%%%

\begin{document}

\maketitle

\frontmatter
\begin{titlepage}

\nonumber
\null \vspace {\stretch{1}}
	\begin{flushright}
%	\begin{verse}
\textit{} \\[5mm]
%	\end{verse}
	\end{flushright}
\vspace{\stretch{2}}\null

\end{titlepage}
\cleardoublepage
\chapter*{Abstract}
\addcontentsline{toc}{chapter}{Abstract}
\chaptermark{Abstract}
\tableofcontents

\mainmatter
\input{main/chapter1/introduzione}
\chapter{Stato dell'Arte}

Lorem ipsum dolor sit amet, consectetur adipiscing elit. Donec pulvinar egestas urna sed mollis. Ut pellentesque libero purus, non euismod turpis gravida mollis. Donec consequat justo vel nisi consequat faucibus. Phasellus nunc velit, maximus et nisl laoreet, ultricies ultrices enim. Aenean et velit et nulla ornare laoreet vel vel nulla. Etiam commodo tempus diam, a ornare mauris. Nulla facilisi. Ut sagittis, est ac tempus pharetra, dui lectus bibendum mauris, sit amet rutrum arcu diam at sem. Nullam blandit eget magna eu ullamcorper. Aenean elementum arcu at sem ultricies finibus. Morbi ullamcorper nulla dolor, eget convallis augue bibendum in. In tincidunt interdum odio ac vehicula. Morbi tristique velit a maximus placerat. Sed quis lacus eleifend, hendrerit libero non, feugiat lectus. Etiam id luctus libero. Cras efficitur dolor id purus lobortis, in facilisis mauris ornare.

Integer pretium dapibus tellus, vehicula sagittis ante euismod ut. Vivamus pretium bibendum purus, posuere consequat quam feugiat in. Duis tellus tellus, scelerisque quis neque a, tempus tincidunt mi. Maecenas consectetur at diam ut blandit. Fusce interdum, ante at commodo mattis, lacus ante bibendum leo, eget ornare arcu eros a dolor. Donec et aliquam risus. Aliquam a dolor at dolor congue pharetra. Vivamus posuere, neque in malesuada facilisis, erat urna fermentum felis, eget mollis nisl eros eu dolor. Sed ultrices aliquam feugiat. Donec id fringilla enim. Nulla molestie mauris id pellentesque volutpat. Vestibulum velit tellus, laoreet ac nunc et, sollicitudin cursus ante. Etiam ultrices eget nisi non porttitor. Nunc tincidunt tellus vel hendrerit dapibus. Etiam lacus ex, semper luctus eros et, pulvinar sagittis lorem.

Nunc laoreet libero massa, aliquam rhoncus quam luctus ut. Ut egestas imperdiet est, quis commodo lorem sollicitudin ut. Cras pellentesque, est nec finibus placerat, massa tortor fermentum nisi, sed accumsan orci tortor id mauris. Cum sociis natoque penatibus et magnis dis parturient montes, nascetur ridiculus mus. Sed ut dolor vestibulum, iaculis nisi vitae, pellentesque urna. Fusce a turpis ultrices, mattis ligula eget, tempor arcu. Mauris ultricies, mauris id tempus scelerisque, lacus sapien interdum odio, non rutrum ante elit quis lectus. Mauris fringilla sem vehicula lectus pretium, id pulvinar leo tincidunt.
\input{main/chapter3/implementazione}
\input{main/chapter4/esperimenti}
\input{main/chapter5/conclusione}
\input{main/appendix/appendice}

\backmatter
\newpage
% \nocite{*}		 % Mostra in bibliografia anche gli oggetti non citati 
\bibliography{back/biblio}{}
\bibliographystyle{plain}

\end{document}     